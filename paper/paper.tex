\documentclass[]{sigchi}

% Arabic page numbers for submission. 
% Remove this line to eliminate page numbers for the camera ready copy
\pagenumbering{arabic}


% Load basic packages
\usepackage{balance}  % to better equalize the last page
\usepackage{graphics} % for EPS, load graphicx instead
\usepackage{times}    % comment if you want LaTeX's default font
\usepackage{url}      % llt: nicely formatted URLs

% llt: Define a global style for URLs, rather that the default one
\makeatletter
\def\url@leostyle{%
  \@ifundefined{selectfont}{\def\UrlFont{\sf}}{\def\UrlFont{\small\bf\ttfamily}}}
\makeatother
\urlstyle{leo}


% To make various LaTeX processors do the right thing with page size.
\def\pprw{8.5in}
\def\pprh{11in}
\special{papersize=\pprw,\pprh}
\setlength{\paperwidth}{\pprw}
\setlength{\paperheight}{\pprh}
\setlength{\pdfpagewidth}{\pprw}
\setlength{\pdfpageheight}{\pprh}

% Make sure hyperref comes last of your loaded packages, 
% to give it a fighting chance of not being over-written, 
% since its job is to redefine many LaTeX commands.
\usepackage[pdftex]{hyperref}
\hypersetup{
pdftitle={SIGCHI Conference Proceedings Format},
pdfauthor={LaTeX},
pdfkeywords={SIGCHI, proceedings, archival format},
bookmarksnumbered,
pdfstartview={FitH},
colorlinks,
citecolor=black,
filecolor=black,
linkcolor=black,
urlcolor=black,
breaklinks=true,
}

% create a shortcut to typeset table headings
\newcommand\tabhead[1]{\small\textbf{#1}}


\title{Croder \\ Bringing the Knowledge of the Crowds into the IDE}
\author{Semih Okur \and Mihai Codoban \and Caius Brindescu \and Kyungho Lee \and Shuo Yuan}
\numberofauthors{5}
\author{
  \alignauthor Semih Okur\\
    \affaddr{University of Illinois at Urbana-Champaign}\\
    \affaddr{Urbana, IL}\\
    \email{okur2@illinois.edu}
  \alignauthor Mihai Codoban\\
    \affaddr{University of Illinois at Urbana-Champaign}\\
    \affaddr{Urbana, IL}\\
    \email{codo@illinois.edu}
  \alignauthor Caius Brindescu\\
    \affaddr{University of Illinois at Urbana-Champaign}\\
    \affaddr{Urbana, IL}\\
    \email{brind@illinois.edu}
   \alignauthor Kyungho Lee\\
    \affaddr{University of Illinois at Urbana-Champaign}\\
    \affaddr{Urbana, IL}\\
    \email{klee141@illinois.edu}
  \alignauthor Shuo Yuan\\
    \affaddr{University of Illinois at Urbana-Champaign}\\
    \affaddr{Urbana, IL}\\
    \email{syuan20@illinois.edu}
}

\begin{document}

\maketitle

\section{Introduction}
\section{Why integrate reviews into the IDE}
\section{Related work}
\section{Selecting code snippets}

\section{Crowds for software development}

One of the main challenges was finding the appropriate crowd to conduct the code review. One of the
first options was the Amazon Mechanical Turk. The main problem with this platform is the lack of
qualified workers. Code Reviewing is a very technical process that requires a large amount of knowledge.
We needed to aim for a platform where we were guaranteed to have the right audience.

Mechanical Turk tasks tend to be very simple and require only minimal knowledge and cognitive
skills. During one experiment we asked a technical question about JavaScript. Out of the 10 hits,
9 were complete in 7 days. This partly shows that the Mechanical Turk platform is ill-suited for tasks
that require a lot of specialized knowledge.

Services such as eLance\footnote{\url{http://www.elance.com}} and oDesk\footnote{\url{http://www.odesk.com}}
employ a crowd to complete programming tasks. But unlike typical crowd sourcing platforms, is it
an offer based system. The requester posts the description for a task and workers bid to complete it.
The requester then chooses a winner and then work on the project can start. This system is not what
we are looking for. We needed a system where you can post your task and workers would select the
task and complete it for a predetermined amount.

\section{Stackexchange code review characteristics}
\section{Interface with StackExchange}
\section{Review management in the IDE}
\section{Crowd sourced peer review creation}
\section{Tying review outcomes to the code}
\section{Preliminary user study}
\section{Sketches}
\section{Conclusion}

\end{document}