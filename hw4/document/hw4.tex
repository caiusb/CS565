\documentclass{article}

\usepackage{fullpage}
\usepackage{graphicx}

\title{Homework 4}
\author{Caius Brindescu \and Mihai Codoban \and Semih Okur \and Shao Yuan \and Kyungho Leoo}

\begin{document}

\maketitle

\begin{figure}
	\begin{center}
		\includegraphics{image.jpg}
	\end{center}
	\caption{The image we used}
	\label{fig:image}
\end{figure}

\begin{figure}
	\includegraphics[width=\textwidth]{wordcloud.png}
	\caption{The word cloud}
	\label{fig:cloud}
\end{figure}

\begin{enumerate}

\item What programming environment did you use? What did you use for generating the word cloud?

We used Java and Eclipse for interacting with the Mechanical Turk and Python for generating the word cloud.

\item Where did the visual design come from and why did you choose it?

We choose an abstract painting from Picasso because we were interested in seeing what concepts different people see in a work of art in which the main message is suggested, hinted upon, rather than revealed upfront.

\item How many workers were involved in producing the final word cloud?

50.

\item What was the total labor cost for producing the word cloud? Don't include the costs related to any earlier experimentation.

\$1.00

\item How long did it take to produce the word cloud (i.e. from the time the program was started to when the word cloud was returned)?

If this includes the time it took for the tasks to be completed, about  a day. If it refers only for the time needed to compute the word cloud, then it was less than 1 second.

\item Diagram the overall workflow or process that you implemented to produce the word cloud.

We got a java word cloud from GitHub (epicdevs/Word-Cloud). It takes a list of strings as input and outputs a BufferedImage which we flushed to file as a PNG.

\item How would you characterize the quality of the first impressions captured by the word cloud?

We got an interesting blend of concepts. Initially we assumed that \emph{music}, \emph{sentiment} or \emph{melange} would be among of the top hits.

\item How did the first impressions returned by the crowd match or deviate from your group's expectations for the design?

Since it was an abstract piece of art, we did not have particular expectations. They could have either converged to a core of concepts, or could have yielded unique results.

\item Describe some ways that might improve the efficiency or outcomes of the task? For example, how could you achieve a higher quality outcome; or how could you reduce the number of workers required, the total labor costs, or the time required to produce an outcome of the same quality?

Insted of using only one image, we could use a batch of different images built around the same concept.

\item From the perspective of the designer who created the design used in the project, how might being able to collect first impressions from potential users benefit the design process?

By getting first impressions, we align ourselves with the global perception. When working in the studio, it is easy to get a skewed perception, to derail from the global train of thought. By getting valuable feedback from the outsied, we align ourselves once more with reality.

\end{enumerate}

\end{document}